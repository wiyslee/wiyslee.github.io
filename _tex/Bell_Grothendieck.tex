\documentclass[psamsfonts]{amsart}

%-------Packages---------
\usepackage{amsmath,amssymb,amsfonts}
\usepackage[all,arc]{xy}
\usepackage{enumerate}
\usepackage{mathrsfs}

%--------Theorem Environments--------
%theoremstyle{plain} --- default
\newtheorem{thm}{Theorem}[section]
\newtheorem{cor}[thm]{Corollary}
\newtheorem{prop}[thm]{Proposition}
\newtheorem{lem}[thm]{Lemma}
\newtheorem{conj}[thm]{Conjecture}
\newtheorem{quest}[thm]{Question}

\theoremstyle{definition}
\newtheorem{defn}[thm]{Definition}
\newtheorem{defns}[thm]{Definitions}
\newtheorem{con}[thm]{Construction}
\newtheorem{exmp}[thm]{Example}
\newtheorem{exmps}[thm]{Examples}
\newtheorem{notn}[thm]{Notation}
\newtheorem{notns}[thm]{Notations}
\newtheorem{addm}[thm]{Addendum}
\newtheorem{exer}[thm]{Exercise}

\theoremstyle{remark}
\newtheorem{rem}[thm]{Remark}
\newtheorem{rems}[thm]{Remarks}
\newtheorem{warn}[thm]{Warning}
\newtheorem{sch}[thm]{Scholium}

\makeatletter
\let\c@equation\c@thm
\makeatother
\numberwithin{equation}{section}

\bibliographystyle{plain}

\DeclareMathOperator{\Tr}{Tr}

%--------Meta Data: Fill in your info------
\title{Tsirelson's Theorem \& Grothendieck's Inequality}

\author{Won I. Lee}

%\date{DEADLINES: Draft AUGUST 18 and Final version AUGUST 29, 2013}

\begin{document}

\maketitle
\section{Bell and CHSH Inequalities}

In response to the infamous EPR paradox, Bell showed that no theory (hidden variable or otherwise) positing locality could produce the correlations theoretically possible in quantum theory. That is, "No physical theory of local hidden variables can ever reproduce all of the predictions of quantum mechanics." (Or a superdeterministic theory with information traveling instantaneously can reproduce it.)

\begin{thm}
	{\bf Bell's Inequality.} If $a,b,c$ are unit vectors along which a spin measurement is made on an entangled pair of spin-1/2 particles in the spin singlet state, such that parallel measurements are completely anticorrelated, then a local hidden variable theory must yields correlations satisfying:
	$$\rho(a,c) - \rho(b,a) - \rho(b,c) \leq 1$$
	where $\rho(a,b) = \langle A(a,\lambda)\cdot B(b,\lambda)\rangle$ and $A, B$ are the measurement devices. 
\end{thm}

While the Bell inequality was a very deep result, it eluded experimental verification for a while because of the requirement of complete anticorrelation (i.e. ideal spin singlet). Moreover, as Bell noted in his original paper, Bell's inequality does not exclude the following options:

1) A local hidden variable theory for a single particle: {\bf entanglement} is entirely necessary.

2) A hidden variable theory that is explicitly nonlocal: that is, Bell's inequality does not exclude hidden variable theories more generally, but requires that the measurement in one detector be allowed to depend on the setting (unit vector $a,b$) of the other detector. This would imply superluminal information transfer.

\begin{thm}
	{\bf CHSH Inequality.} In the same setup as Bell, let $a,a'$ be unit vectors along which measurement is made in the $A$ detector, and $b,b'$ for the $B$ detector. Then, a local hidden variable theory implies:
	$$|\rho(a,b) - \rho(a,b') + \rho(a',b) + \rho(a',b')| \leq 2$$
\end{thm}

While the CHSH inequality improves upon Bell's for the purposes of experimental physics (since no perfect spin singlet is necessary), it is still not very general. It turns out that Bell's inequalities can be represented as {\bf linear inequalities on observables} rather than depending on particular states. In particular, the CHSH inequality generalizes as:

\begin{thm}
	{\bf CHSH (Operators).} For arbitrary Hermitian elements of a C*-algebra $A_1,A_2,B_1,B_2$ such that $\|A_i\|,\|B_j\| \leq 1$ and $[A_k,B_j] = 0$,
	$$A_1B_1 + A_1B_2 + A_2B_1 - A_2B_2 \leq 2\sqrt{2}\cdot I$$
\end{thm}

which follows from a more general relation of:
$$(A_1B_1+A_1B_2 + A_2B_1 - A_2B_2)^2 \leq 4\cdot I - [A_1,A_2]\cdot[B_1,B_2]$$

\section{Classical and Quantum Correlation Matrices}

We can classify entire spaces of correlation matrices based on whether they can be realized by a "classical" (i.e. local hidden variable) theory or a "quantum" theory. The context is as follows: two detectors make measurements according to $a_1, \dots, a_n$ and $b_1, \dots, b_m$ unit vectors on two entangled spin-1/2 particles. If we again define $\rho(a_j,b_k) \equiv \langle A(a_j,\lambda)\cdot B(b_k,\lambda)\rangle$, then the corresponding correlation matrix is:
$$C(\lambda) = [\rho(a_j,b_k)]$$
and any classical correlation matrix is a convex combination: $C = \sum_{\lambda} p_{\lambda} C(\lambda)$ or $C = \int \rho(\lambda) C(\lambda) d\lambda$. Thus:
$$\Lambda_C \equiv \text{ Classical correlation matrices } = ch(C(\lambda))$$

In the quantum case, the measurements correspond to observables $A_1, \dots, A_n$, $B_1, \dots, B_m$ with eigenvalues $\pm 1$. Then the correlation matrix is given by:
$$\rho(A_i,B_j) = \langle \psi|A_j\otimes B_k|\psi\rangle$$
and
$$\Lambda_Q \equiv \text{ Quantum correlation matrices } = \left\{ C = [\langle \psi|A_j\otimes B_k|\psi\rangle], \text{ for some } A_i, B_j, \text{ state } |\psi\rangle\right\}$$
consists of matrices that can be generated by suitable sets of observables.

In this formulation, we can also characterize $\Lambda_C \subset \Lambda_Q$ with the same condition, except that all operators $A_i, B_j$ commute.

\section{Tsirelson's Theorem}

Though proved in greater generality, Tsirelson provided a clear characterization of quantum correlation matrices:

\begin{thm}
	{\bf Tsirelson.} $C$ is a quantum correlation matrix iff there exist $u_1, \dots, u_n, v_1, \dots, v_m \in \mathbb{R}^{n+m}$ with $\|u_j\| = \|v_k\| = 1$ for all $j,k$, such that:
	$$[C]_{jk} = u_j \cdot v_k$$
\end{thm}

This decomposition is not just useful for a clearer characterization of quantum correlation matrices, but also because it allows one to employ Grothendieck's inequality to bound the values a quantum matrix can take. 

\begin{thm}
	{\bf Grothendieck.} For an arbitrary $n\times n$ matrix $[\alpha_{ij}]$, and $u_1, \dots, u_n, v_1, \dots, v_m \in H$ for Hilbert space $H$, we have:
	$$\left| \sum \alpha_{ij} u_i \cdot v_j \right| \leq K_G \sup_i \|u_i\| \cdot \sup_j \|v_j\|$$ 
\end{thm}

Thus, in this formulation we can see that for any quantum correlation matrix $[C]_{ij} = u_j \cdot v_k$:

$$\left| \sum \alpha_{ij}  [C]_{ij}\right|  =\left| \sum \alpha_{ij}  u_i \cdot v_j\right| \leq K_G \sup_i \|u_i\| \cdot \sup_j \|v_j\| \leq K_G \sup_{C' \in \Lambda_C} \sum \alpha_{ij} [C']_{ij}$$

Thus, we have the following result:
$$\Lambda_Q \subset K_G \Lambda_C$$

\begin{thebibliography}{9}

\bibitem{tsirelson85}
B. S. Tsirelson, “Quantum analogs of the Bell inequalities. The case of two spatially separated domains”,
J. Math. Sci. 36(3), 557–570 (1987). Translated from Russian: Zapiski LOMI 142, 174-194 (1985).

\bibitem{resume}
A. Grothendieck, “R´esum´e de la th´eorie m´etrique des produits tensoriels topologiques” (French), Bol. Soc.
Mat. Sao Paulo 8, 1–79 (1953).

\bibitem{GT}
G. Pisier, "Grothendieck's theorem, past and present", arXiv.

\end{thebibliography}

\end{document}

